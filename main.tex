\documentclass{article}
\usepackage{amsmath}

\begin{document}

\title{Ein Einfaches LaTeX-Dokument mit Mathematischen Operationen 2}
\author{Shadi Zumor}
\date{\today}
\maketitle

\section{Einführung}
Dies ist ein einfaches LaTeX-Dokument, das einige mathematische Operationen enthält.

\section{Mathematische Operationen}

Hier sind einige Beispiele für mathematische Ausdrücke:

\subsection{Grundrechenarten}
Hier sind einige Grundrechenarten:

\begin{align*}
a + b &= c \\
d - e &= f \\
g \times h &= i \\
j \div k &= l
\end{align*}

\subsection{Brüche}
Hier ist ein Beispiel für einen Bruch:

\begin{equation}
\frac{a}{b} = c
\end{equation}

\subsection{Quadratische Gleichung}
Die quadratische Gleichung lautet:

\begin{equation}
ax^2 + bx + c = 0
\end{equation}

Die Lösung der quadratischen Gleichung ist:

\begin{equation}
x = \frac{-b \pm \sqrt{b^2 - 4ac}}{2a}
\end{equation}

\subsection{Summation}
Hier ist ein Beispiel für eine Summation:

\begin{equation}
\sum_{n=1}^{\infty} \frac{1}{n^2} = \frac{\pi^2}{6}
\end{equation}

\subsection{Integration}
Hier ist ein Beispiel für ein Integral:

\begin{equation}
\int_{0}^{1} x^2 \, dx = \frac{1}{3}
\end{equation}

\end{document}
